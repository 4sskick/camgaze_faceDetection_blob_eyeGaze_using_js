

\documentclass[annual]{acmsiggraph}
\usepackage{algorithm}
\usepackage[noend]{algorithmic}

\TOGonlineid{}
\TOGvolume{}
\TOGnumber{}
\TOGarticleDOI{}
\TOGprojectURL{}
\TOGvideoURL{}
\TOGdataURL{}
\TOGcodeURL{}

\newcommand{\Acronym}[1]{\ensuremath{{\small{\texttt{#1}}}}}
\newcommand{\Name}{\Acronym{Camgaze.js}}
\newcommand{\False}{\Constant{false}}
\newcommand{\True}{\Constant{true}}
\newcommand{\Symbol}[1]{\ensuremath{\mathcal{#1}}}
\newcommand{\Function}[1]{\ensuremath{{\small \textsc{#1}}}}
\newcommand{\Constant}[1]{\ensuremath{\small{\texttt{#1}}}}
\newcommand{\Var}[1]{\ensuremath{{\small{\textsl{#1}}}}}

\title{Camgaze.js : Mobile Eye Tracking and Gaze Prediction in JavaScript}
\author{Alex Wallar \\ Christian Poellabauer \\ Patrick Flynn}
\pdfauthor{Alex Wallar}
\date{\today}

\begin{document}

\maketitle

\begin{abstract}
\end{abstract}

\section{Introduction}

\section{Motivation}

\section{Related Works}

\section{Implementation}

$\Name$ goes through two steps in order to predict the gaze direction. 
Firstly, $\Name$ detects each pupil. It then uses the pupils deviation 
from a unique point on the face to determine the gaze metric, \Symbol{G}. 
This metric needs to be calibrated in order for there to be a mapping 
from \Symbol{G} to a point on the screen. Once this gaze metric has been 
calibrated, $\Name$ should be able to interpolate area of the 
screen the user is looking at. A high level description of the 
algortihm is shown below.

\begin{algorithm}
\caption{Pseudocode for $\Name$}
\label{algo:Main}
\begin{algorithmic}[1]
\setcounter{ALC@line}{0}

\vspace*{1mm}

\STATE $\Symbol{F} \leftarrow \Function{InitGazeMapping}()$

\WHILE{$\Function{StillCalibrating()} == \True$}
\STATE $P_{list} \leftarrow \Function{DetectPupils()}$
\STATE $\Symbol{G} \leftarrow \Function{DetermineGazeMetric}(P_{list})$
\STATE $\Symbol{F} \leftarrow \Function{Calibrate}(\Symbol{G}, \Symbol{F})$
\ENDWHILE

\WHILE{$\Function{SessionFinished()} == \False$}
\STATE $P_{list} \leftarrow \Function{DetectPupils()}$
\STATE $\Symbol{G} \leftarrow \Function{DetermineGazeMetric}(P_{list})$
\STATE $\Function{ProjectGazeOntoScreen}(\Symbol{F}(\Symbol{G}))$
\ENDWHILE

\end{algorithmic}
\end{algorithm}

\subsection{Pupil Detection}

Detecting the pupils enables $\Name$ to determine the gaze direction.
Pupil detection in this approach is aimed to be fast in order to 
be deployable onto mobile devices. Firstly, the eyes are detected
using the Viola-Jones Object Detection Framework \cite{Viola01}. 

\subsection{Determining the Gaze Metric}

\subsection{Calibration}

\section{Testing}

\section{Applications}

\section{Discussion}

\bibliographystyle{acmsiggraph}
\bibliography{paper}

\end{document}
